\documentclass[a4paper]{article}

\usepackage[T1, T2A]{fontenc}
\usepackage[utf8]{inputenc}
\usepackage[english, russian]{babel}
\usepackage{tempora}
\usepackage[11pt]{extsizes}
\usepackage[top=2cm, bottom=2cm, left=1.5cm, right=1.5cm]{geometry}
\usepackage{enumitem}
\usepackage{setspace}
\usepackage{ifthen}
\usepackage[normalem]{ulem}
\useunder{\uline}{\ulined}{}

\pagestyle{empty}
\makeatletter
\newcounter{i}
\newcommand{\PrintEmptyLines}[1]{\setcounter{i}{1}\@whilenum\value{i}<#1\do{\stepcounter{i}\EmptyLine\\} \ifnum #1>0 {\EmptyLine}\fi}
\makeatother
\newcommand\arulefill[1]{{\expandafter \ulined #1 \hfill}}
\newcommand{\EmptyLine}{\null\arulefill{}}
\newcommand{\mrule}[1]{\rule[-2.8pt]{#1}{.4pt}}
\newcounter{EmptyLines1}
\newcounter{EmptyLines2}
\newcounter{EmptyLines3}
\newcounter{EmptyLines10}
\newcounter{EmptyLines11}
\newcounter{EmptyLines12}

\linespread{0.9}

%%%%%%%%%%%%%%%%%%%%%%%%%%%%%%
%% НИЖЕ ПОЛЯ ДЛЯ ЗАПОЛНЕНИЯ %%
%%%%%%%%%%%%%%%%%%%%%%%%%%%%%%

\newcommand{\LabNumber}{00}  % номер ЛР
\newcommand{\Discipline}{Практикум на ЭВМ}  % название дисциплины (курса)
\newcommand{\StudentGroup}{М8О-100Б-20}  % группа студента
\newcommand{\StudentName}{Иванов Иван Иванович}  % имя студента
\newcommand{\StudentNumber}{00}  % номер по списку
\newcommand{\StudentContacts}{ya@yandex.ru}  % контакты студента
\newcommand{\DateOfCompletion}{<< 1 >> января 2022 г.}  % дата выполнения работы
\newcommand{\Teacher}{ст. преп. каф. 806 Дубинин А.В.}  % преподаватель
\newcommand{\NumberVariant}{ }  % номер варианта

\newcommand{\Texti}{Тема}  % Тема
\newcommand{\Textii}{Цель работы}  % Цель работы
\newcommand{\Textiii}{Задание}  % Задание 
\newcommand{\Textvi}{Идея}  % Идея, метод, алгоритм
\newcommand{\Textvii}{Сценарий выполнения работы}  % Сценарий выполнения работы
\newcommand{\Textviii}{}  % Распечатка протокола
\newcommand{\Textx}{}  % Замечания автора
\newcommand{\Textxi}{Выводы}  % Выводы
\newcommand{\Textxii}{}  % Устранение недочётов

% Количество пустых подчеркнутых полей для:

\setcounter{EmptyLines1}{1}  % Тема
\setcounter{EmptyLines2}{2}  % Цель работы
\setcounter{EmptyLines3}{3}  % Задание
\setcounter{EmptyLines10}{3}  % Замечания автора
\setcounter{EmptyLines11}{10}  % Выводы
\setcounter{EmptyLines12}{5}  % Устранение недочётов


\begin{document}
\begin{minipage}{0.1\textwidth}
    \fbox{\rule[2.1cm]{1.7cm}{0pt}}
    \vspace{3.5cm}
\end{minipage}
\begin{minipage}{0.858\textwidth}
    \begin{center}
        \Large\textbf{Отчет по лабораторной работе № {\LabNumber} по курсу \uline\Discipline}
    \end{center}
    
    \begin{doublespace}
        \vbox{\hfill\vbox{%
        \hbox{Студент группы \uline{\StudentGroup \ \StudentName}, № по списку \uline\StudentNumber}%
        \hbox{Контакты www, e-mail, icq, skype \uline{\StudentContacts}}
        }}
    \end{doublespace}
    \begin{doublespace}
        \vbox{\hfill\vbox{%
        \hbox{Работа выполнена: \DateOfCompletion}%
        \hbox{Преподаватель: \uline{\Teacher}}%
        \hbox{Входной контроль знаний с оценкой \mrule{3.3cm}}%
        \hbox{Отчет сдан << \hspace{0.3cm} >> \mrule{1.5cm} 202 \mrule{.3cm} г., итоговая оценка \mrule{.5cm}}
        }}
    \end{doublespace}
    \hfill\hbox{Подпись преподавателя \mrule{3cm}}
\end{minipage}


\begin{enumerate}[label=\textbf{\arabic*}.]

\item \textbf{Тема:} \arulefill{\Texti} \\ 
\PrintEmptyLines{\value{EmptyLines1}}

\item \textbf{Цель работы:} \arulefill{\Textii} \\
\PrintEmptyLines{\value{EmptyLines2}}

\item \textbf{Задание} (\textit{вариант № \NumberVariant}): \arulefill{\Textiii} \\
\PrintEmptyLines{\value{EmptyLines3}}

\item \textbf{Оборудование} (\textit{лабораторное}):\\
ЭВМ \EmptyLine, процессор \EmptyLine, имя узла сети \EmptyLine \ c ОП \EmptyLine \ Мб,\\
НМД \EmptyLine \ Мб. Терминал \EmptyLine \ адрес \EmptyLine. Принтер \EmptyLine \\
Другие устройства \EmptyLine \\ \EmptyLine \\\\
\textit{Оборудование ПЭВМ студента, если использовалось:}\\
Процессор \EmptyLine \ с ОП \EmptyLine \ Мб, НМД \EmptyLine \ Мб. Монитор \EmptyLine \\
Другие устройства \EmptyLine \\ \EmptyLine

\item \textbf{Программное обеспечение} (\textit{лабораторное}):\\
Операционная система семейства \EmptyLine, наименование \EmptyLine \ версия \EmptyLine \\
интерпретатор команд \EmptyLine \ версия \EmptyLine \\
Система программирования \EmptyLine \ версия \EmptyLine \\
Редактор текстов \EmptyLine \ версия \EmptyLine \\
Утилиты операционной системы \EmptyLine \\ \EmptyLine \\
Прикладные системы и программы \EmptyLine \\
Местонахождение и имена файлов программ и данных \EmptyLine \\ \EmptyLine \\\\
\textit{Программное обеспечение ЭВМ студента, если использовалось:}\\
Операционная система семейства \EmptyLine, наименование \EmptyLine \ версия \EmptyLine \\
интерпретатор команд \EmptyLine \ версия \EmptyLine \\
Система программирования \EmptyLine \ версия \EmptyLine \\
Редактор текстов \EmptyLine \ версия \EmptyLine \\
Утилиты операционной системы \EmptyLine \\ \EmptyLine \\
Прикладные системы и программы \EmptyLine \\
Местонахождение и имена файлов программ и данных на домашнем компьютере \EmptyLine \\ \EmptyLine

\item 
\begin{minipage}[t][0.45\textheight]{.95\textwidth}
\textbf{Идея, метод, алгоритм} {\footnotesize решение задачи (в формах: словесной, псевдокода, графической [блок-схема, диаграмма, рисунок, таблица] или формальные спецификации с пред- и постусловиями)} \\\\ \Textvi 
\end{minipage}

\item 
\begin{minipage}[t][0.45\textheight]{.95\textwidth}
\textbf{Сценарий выполнения работы} {\footnotesize (план работы, первоначальный текст программы в черновике [можно на отдельном листе] и тесты либо соображения по тестированию)} \\\\ \Textvii
\end{minipage}

\textit{Пункты 1-7 отчета составляются строго до начала лабораторной работы.} \\

\hfill\hbox{\textit{Допущен к выполнению работы.} \textbf{Подпись преподавателя \mrule{4cm}}}
\newpage

\item \textbf{Распечатка протокола} {\footnotesize (подклеить листинг окончательного варианта программы с тестовыми примерами, подписанный преподавателем)} \\
\newpage

\item \textbf{Дневник отладки} {\footnotesize должен содержать дату и время сеансов отладки и основные события (ошибки в сценарии и программе, нестандартные ситуации) и краткие комментарии к ним. В дневнике отладки приводятся сведения об использовании ЭВМ, существенном участии преподавателя и других лиц в написании и отладке программы.} \\ \Textviii
\begin{tabular}[t]{|c|c|c|c|c|c|c|}
\hline
№ & \begin{tabular}[t]{c} Лаб. \\ или \\ дом. \end{tabular} & Дата & Время & \hspace{.7cm} Событие \hspace{.7cm} & Действие по исправлению & \hspace{.7cm} Примечание \hspace{.7cm} \\
\hline
\begin{minipage}[t][0.45\textheight]{0.01\textwidth}\end{minipage} & & & & & &\\
\hline
\end{tabular}

\item \textbf{Замечания автора} {\footnotesize по существу работы:} \arulefill{\Textx} \\
\PrintEmptyLines{\value{EmptyLines10}}

\item \textbf{Выводы:} \arulefill{\Textxi} \\
\PrintEmptyLines{\value{EmptyLines11}}\\

 Недочёты при выполнении задания могут быть устранены следующим образом: \Textxii
\PrintEmptyLines{\value{EmptyLines12}}\\
 \begin{flushright}
Подпись студента \mrule{4cm}
\end{flushright}

\end{enumerate}
\end{document}
